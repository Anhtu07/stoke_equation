\documentclass[14pt]{extreport}
\usepackage[utf8]{vietnam}
%\usepackage{type1cm}
\usepackage[left=3.50cm, right=2.00cm, top=3.50cm, bottom=3.00cm]{geometry}
%\usepackage[left=3.0cm, right=1.50cm, top=3.00cm, bottom=2.50cm]{geometry}
\usepackage{graphicx}
\usepackage{mathrsfs} 
\usepackage{amsfonts}
\usepackage{longtable}
\usepackage[intlimits]{amsmath}
\usepackage{array}
\usepackage{amsxtra,amssymb,latexsym,amscd,amsthm}
\usepackage{hyperref}
\newtheorem{theorem}{\MakeUppercase{K}ết quả}[section]
% khoảng cách dòng 1.5 lines (như trong MS Word)
\renewcommand{\baselinestretch}{1.5}

%——————–
\begin{document}
%tao khung
\newcommand{\Khung}[2]{
\begin{tabular}{|l|}
\hline\rule[-2ex]{0pt}{5.5ex}
\parbox{#1}{#2}\\
\hline
\end{tabular}
}

\Khung{.92\textwidth}{

\begin{center}
\normalsize
\textbf{TRƯỜNG ĐẠI HỌC BÁCH KHOA HÀ NỘI}\\
\normalsize
\textbf{VIỆN TOÁN ỨNG DỤNG VÀ TIN HỌC}\\
\textbf{------------------------------------------------------}\\[0.4cm]
\includegraphics[scale=.2]{logobkdentrang}\\[1.2cm]
\textbf{{\large PHƯƠNG PHÁP PHẦN TỬ HỮU HẠN GIẢI PHƯƠNG TRÌNH STOKES}}\\
\end{center}
\begin{flushleft}
\vspace{1.3cm}
\hspace{1.5cm} \textbf{ Giảng viên hướng dẫn:{ TS. Phan Xuân Thành }}\\[0.2cm]
\hspace{1.5cm} \textbf{ Nhóm Sinh viên thực hiện:}\\[0.2cm]
\hspace{5cm}\textbf{Nguyễn Anh Tú}\\[0.2cm]
\hspace{5cm}\textbf{Phạm Anh Tuấn}\\[0.2cm]
\hspace{1.5cm} \textbf{ Lớp:\hspace{2cm}{ KSTN Toán Tin K60}}\\
\end{flushleft}

\begin{center}
\textbf{{\small HÀ NỘI - 1/2019}}\\
\end{center}
 }
\thispagestyle{empty}
\newpage

\tableofcontents
\newpage

%\listoffigures

\newpage
{\huge \textbf{Lời mở đầu}} \\

Phương trình Navier-Stokes có ý nghĩa quan trọng trong vật lý, nó miêu tả dòng chảy của các chất lỏng và khi, còn gọi chất lưu. Phương trình được thiết lập trên cơ sở biến thiên động lượng trong những thể tích vô cùng nhỏ của chất lưu, gồm tổng của các lực nhớ, biến đổi áp suất, trọng lực và các ngoại lực khác tác động lên chất lưu. Trong phạm vi của bài báo cáo này, nhóm thực hiện xin được trình bày một trường đặc biệt của phương vật lý chuyển động của chất lỏng, phương trình Stokes trong không gian 2 chiều. \\

Bên cạnh việc thực hiện chương trình giải bài toán Stokes bằng phương pháp phần tử hữu hạn, trong bài báo cáo này còn trình bày cơ sở lý thuyết xây dựng nên bài toán yếu, tính đặt chỉnh bài toán Stokes. \\


\newpage
\chapter{Kiến thức cơ sở}
\section{Tích vô hướng}

$V$- Không gian vectơ trên $R$, tích vô hướng $(u,v)_V: V\times V \rightarrow \mathbb{R}$ có các tính chất: $\forall u,v,w \in V$
\begin{enumerate}
\item $(u,v)=(v,u)$ 
\item $(u,v+w)=(u,v)+(u,w)$
\item $(u, kv) = k(u,v)$
\item $(u,u)\geq 0$
\item $(u,u)=0 \Leftrightarrow  u \equiv \theta  \in V$
\end{enumerate}

Bất đẳng thức:
\begin{align*}
(u,v)^2 \leq (u,u).(v,v) \quad \forall u,v \in V
\end{align*}

\section{Không gian Hilbert}
Không gian vectơ $V$ ($V$ là không gian đủ) với tích vô hướng $(.,.)$ gọi là không gian Hilbert.\\
Chuẩn trong không gian Hilbert:
$$ \left \| u \right \|_V = \sqrt{(u,u)_V}$$\\
Bất đẳng thức của tích vô hướng được chuyển về:
\begin{align*}
|(u,v)| \leq \left \| u \right \|_V.\left \| v \right \|_V
\end{align*}

\section{Đạo hàm yếu}
Xét hàm $u(x), x \in [a,b]$, $u$ khả vi: tồn tại $u'(x)=\underset{h \rightarrow  0}{lim}\frac{u(x+h)-u(x)}{h}$
Đạo hàm yếu của hàm $u(x)$ là hàm $g(x) = \frac{\partial u}{\partial x_k}$ sao cho:
\begin{align*}
\int_{a}^{b}g(x)v(x)dx = - \int_a^bu(x)\frac{\partial v}{\partial x_k}dx  \quad  \forall v \in D(a,b)
\end{align*}
Ta định nghĩa tương tự với đạo hàm cấp $|\alpha|$:
\begin{align*}
D^{\alpha}u = \frac{\partial^{|\alpha|}u }{\partial^{\alpha_1}x_1\partial^{\alpha_2}x_2 ...\partial^{\alpha_n}x_n  }
\end{align*}

\section{Một số không gian hay dùng}
Không gian $L_2(\Omega)$: là không gian các hàm $u(x)$ thoả mãn:
\begin{align*}
\int_{\Omega}\left |  u(x)\right |^{2}dx < +\infty 
\end{align*}
Chuẩn và tích vô hướng cho không gian $L_2(\Omega)$:
\begin{align*}
\left \| u \right \|_{L_2(\Omega)} &= \left ( \int_{\Omega}\left |  u\right |^{2}dx \right )^{\frac{1}{2}}\\
(u,v)_{L_2(\Omega)} &= \int_{\Omega}u(x)v(x)dx
\end{align*}
Do không gian $L_2(\Omega)$ là không gian Hilbert nên:
\begin{align*}
\left ( \int_{\Omega}u(x)v(x)dx \right )^2 &\leq \int_{\Omega}\left |  u\right |^{2}dx. \int_{\Omega}\left |  v\right |^{2}dx
\end{align*}
Tương tự cho không gian $L_p(\Omega)$: là không gian các hàm $u(x)$ thoả mãn $\int_{\Omega}\left |  u\right |^{p}dx < +\infty $
\begin{align*}
\left \| u \right \|_{L_p(\Omega)} &= \left ( \int_{\Omega}\left |  u\right |^{p}dx \right )^{\frac{1}{p}}\\
\left | \int_{\Omega}u(x)v(x)dx \right | &\leq \left ( \int_{\Omega}\left |  u\right |^{p}dx \right )^{\frac{1}{p}}.\left ( \int_{\Omega}\left |  v\right |^{q}dx \right )^{\frac{1}{q}}
\end{align*}
với $\frac{1}{p}+\frac{1}{q}=1$\\
Không gian $W_p^m(\Omega)$:  $H^m(\Omega) \equiv  W^m_2(\Omega)$
\begin{align*}
W_m^p(\Omega) = \{ u | D^{\alpha}u \in L_p(\Omega) \forall \alpha : |\alpha|\leq m, m \in \mathbb{N}\}
\end{align*}

Chuẩn và tích vô hướng:
\begin{align*}
(u,v)_{H^m(\Omega)} &= \sum_{|\alpha|\leq m}(D^{\alpha}u, D^{\alpha}v)_{L_2(\Omega)}\\
(u,v)_{L_2(\Omega)} &= \int_{\Omega}u(x)v(x)dx\\
\left \|  u\right \|_{H^m(\Omega)} &= \sqrt{(u,u)_{H^m(\Omega)}}
\end{align*}

Không gian $H^m(\Omega)$: là không gian Hilbert: $H^0(\Omega) \equiv  L_2(\Omega)$:
\begin{align*}
\left \| u \right \|_{H^1(\Omega)}= \sqrt{\int_{\Omega}u^2(x)dx+\sum_{k=1}^n\int_{\Omega}\left ( \frac{\partial u}{\partial x_k} \right )^2dx}
\end{align*}


\section{Bất đẳng thức Poincare}
Xét $\Omega \subset \mathbb{R}^n$, $\Omega$ bị chặn. Với $\forall  u \in H^1(\Omega)$, tồn tại hằng số $C$ (phụ thuộc $\Omega$) sao cho:
\begin{align*}
\int_{\Omega}u^2(x)dx \leq C\left (  \sum_{k=1}^{n}\int_{\Omega}u_{x_k}^{2}dx+\oint_{\Gamma}u^{2}(x)dS\right )
\end{align*}

Trong các phần sau, để cho ngắn gọn, khi miền xác định đã rõ, ta kí hiệu
$$||u||_0 := ||u||_{L_2(\Omega)}$$
$$||u||_1 := ||u||_{H_1^0(\Omega)}$$

\chapter{Hệ phương trình Stokes}
{\normalsize
Trong bài báo cáo này, chúng ta quan tâm đến việc giải gần đúng nghiệm của hệ phương trình Stokes đối với bài toán dòng chảy không nhớt. Ở dạng đơn giản nhất, hệ phương trình Stokes trong trường hợp 2 chiều trên miền $\Omega$ được phát biểu như sau:
$$
\begin{cases}
-\Delta u + grad p \ = \ f \\
div u = 0 \\
u|_{\Gamma} = g
\end{cases}
$$
trong đó $u$ là hàm vector khả vi hai lần trên miền $\Omega$ và $\Gamma = \partial \Omega$
\section{Xây dựng bài toàn yếu}
Đặt $e_{ij}(u) =  \frac{1}{2} ( \frac{\partial u_i}{\partial x_j} + \frac{\partial u_j}{\partial x_i})$ $\forall i,j = 1,2$. Ta có
\begin{equation} \label{eq1}
\begin{split}
\sum_{j=1}^{2} \frac{\partial}{\partial x_j} e_{kj}(u) & = \sum_{j-1}^2 \frac{1}{2} \left(\frac{\partial^2 u_k}{\partial x_j^2} + \frac{\partial^2 u_j}{\partial x_j \partial x_k}\right) \\
 & = \frac{1}{2} \sum_{j=1}^2 \frac{\partial^2 u_k}{\partial x_j^2} + \frac{1}{2} \sum_{j=2}^2 \frac{\partial^2 u_j}{\partial x_j \partial x_k} \\
 & = \frac{1}{2} \sum_{j=1}^2 \frac{\partial^2 u_k}{\partial x_j^2} + \frac{1}{2} \frac{\partial}{\partial x_k} \left(\sum_{j=1}^2 \frac{\partial u_j}{\partial x_j} \right) \\
 & = \frac{1}{2} \Delta u_k(x) + \frac{1}{2} \frac{\partial}{\partial x_k} div \ u(x) \ \ \forall \ k = 1,2
\end{split}
\end{equation}
Nhân thêm hàm thử $v_k(x)$
\begin{equation} \label{eq2}
\begin{split}
\sum_{j=1}^{2} \frac{\partial}{\partial x_j} e_{kj}(u) v_k(x) & = v_k(x) \sum_{j=1}^2 \frac{\partial}{\partial x_j} e_{kj} (u) + \sum_{j=1}^2 e_{kj}(u) \frac{\partial v_k(x)}{\partial x_j} \\
 & = \frac{1}{2} v_k(x) \Delta u_k + \frac{1}{2} v_k \frac{\partial}{\partial x_k} div u(x) + \sum_{j=1}^2 e_{kj}(u) \frac{\partial v_k(x)}{\partial x_j}
\end{split}
\end{equation}
Lấy tổng trên $k = 1,2$ ta có
\begin{equation} \label{eq3}
\begin{split}
\sum_{k,j=1}^{2} \frac{\partial}{\partial x_j} e_{kj}(u) v_k(x) & = \frac{1}{2} \sum_{k=1}^2 \left[ \Delta u_k + \frac{\partial}{\partial x_k} div \ u(x) \right] v_k(x) + \sum_{i, j=1}^2 e_{kj}(u) e_{kj}(v)
\end{split}
\end{equation}
Chuyển $\Delta u_k$ sang vế trái, ta thu được
\begin{equation} \label{eq4}
\begin{split}
- \sum_{k=1}^2 v_k(x) \Delta u_k(x) & = 2 \sum_{k,j=1}^2 e_{kj}(u) e_kj(v) - 2\sum_{i,j=1}^2 \frac{\partial}{\partial x_j} (e_{kj}(u) v_k(x)) \\
& + \sum_{k=1}^2 v_k(x) \frac{\partial}{\partial x_k} div \ u(x)
\end{split}
\end{equation}

Nhân hai vế của phương trình Stoke $- \Delta u + \nabla p(x) = f(x)$ với hàm thử $v(x)$ và lấy tích phân trên miền $\Omega$, ta có
\begin{equation} \label{eq5}
\begin{split}
\int_{\Omega} f_k(x) v_k(x) dx & = - \int_{\Omega} v_k(x) \Delta u_k(x) dx + \int_{\Omega} \frac{\partial p(x)}{\partial x_k} v_k(x) dx \\
 & = - \int_{\Omega} v_k(x) \Delta u_k(x) dx - \int_{\Omega} p(x) \frac{\partial v_k}{\partial x_k} dx + \int_{\Gamma} p(x) v_k(x) n_k(x) dS  \\
& \forall k = 1,2
\end{split}
\end{equation}

Lấy tổng theo $k = 1,2$, và thay vào phương trình $(4)$, ta thu được
\begin{equation} \label{eq6}
\begin{split}
\int_{\Omega} v(x)^T f(x) dx & = 2 \sum_{k, j = 1}^2 \int_{\Omega} e_kj(u, x) e_{kj} (v,x) - 2 \sum_{k,j=1}^2 \int_{\Omega} \frac{\partial}{\partial x_j} \left[ e_{kj}(u, x) v_k(x) \right] \\
& + \sum_{k=1}^2 \int_{\Omega} v_k(x) \frac{\partial}{\partial x_k} div \ u(x) dx - \int_{\Omega} p(x) div \ v(x) dx \\
& + \sum_{k =1}^2 \int_{\Gamma} v_k(x) n_k(x) dS 
\end{split}
\end{equation}

Áp dụng công thức tích phân từng phần ta thu được công thức Green cho bài toán Stoke
\begin{equation} \label{eq7}
\begin{split}
\int_{\Omega} v(x)^T f(x) dx & = 2 \sum_{k, j = 1}^2 \int_{\Omega} e_kj(u, x) e_{kj} (v,x) + 2 \sum_{k,j=1}^2 \int_{\Gamma} e_{kj}(u, x) v_k(x) n_j(x) dS \\
& - \int_{\Omega} div \ u div \ v dx + \int_{\Gamma} div \ u n^T(x) v(x) dS \\
& - \int_{\Omega} p(x) div \ v dx + \sum_{k =1}^2 \int_{\Gamma} p(x) v_k(x) n_k(x) dS 
\end{split}
\end{equation}

Chọn hàm thử $v \in [ H_0^1(\Omega) ]^2$, vì $u$ bằng 0 trên biên $\Gamma$ và $div \ u = 0$, nên ta có

\begin{equation} \label{eq8}
\begin{split}
2 \sum_{k,j = 1}^2 \int_{\Omega} e_{kj}(u,x) e_{kj} (v,x) dx - \int_{\Omega} p div \ v dx = \int_{\Omega} v^T(x) f(x) dx \ \forall \ v \in [ H_0^1(\Omega) ]^2
\end{split}
\end{equation}

Đặt :
$$\alpha(u, v) = 2 \sum_{k,j = 1}^2 \int_{\Omega} e_{kj}(u,x) e_{kj} (v,x)$$
$$\beta(v, q) = \int_{\Omega} q(x) \ div \ v dx$$
Ta thấy rằng nghiệm của $p(x)$ có thể sai khác nhau một hằng số, do đó đặt $Q = \{q(x) |\in L_2(\Omega) \ and \ \int_{\Omega} q(x) dx = 0 \}$. Ta có bài toán yếu cho phương trình Stoke: \\
Tìm hàm $u \in  [ H_0^1(\Omega) ]^2$ và hàm $p \in Q$ sao cho
$$
\begin{cases}
\alpha(u,v) - \beta(v, p) = <f, v> \ \forall \ v \in [ H_0^1(\Omega) ]^2 \\
\beta(u, q) = 0 \ \forall \ q \in Q
\end{cases}
$$

\section{Tính đặt chỉnh}
Ở dạng tổng quát, bài toán Stoke được gọi là bài toán hỗn hợp (mixed problem) được miêu tả như sau [4]:
\begin{center}
Tìm $u \in V$  và $p \in M'$  thỏa mãn
\end{center}
$$
\begin{cases}
Au + B^*p = f, \\
Bu = g
\end{cases}
$$
trong đó $V$ và $M$ là hai không gian Banach, $A: V \rightarrow V'$ và $B: V \rightarrow M$ là hai ánh xạ tuyến tinh bị chặn (ở đây, ta kí hiệu $X'$ là không gian đối ngẫu của $X$). $B*: M' \rightarrow V'$ là toán tử đối ngẫu của $B$ (adjoint operator), $f \in V'$ và $g \in M$. Mục đích của phần này là ta sẽ miêu tả tính đặt chỉnh (well-posedness) của hệ phương trình trên, từ đó liên hệ với bài toán Stoke. \\

Gọi $ker(B)$ là nhân của toán tử $B$ và $A_{\pi}: ker(B) \rightarrow ker(B')$ sao cho $<A_{\pi}v, w>_{V', V} = <Av, w>_{V', V}$ với mọi $v, w \in ker(B)$, khi đó $A_{\pi} = J_B^*AJ_B$ với $J_B$ là đơn ánh từ $ker(B)$ vào $V$ và $J_B^*$ là toán tử đối ngẫu của $J_B$. Theo [4], ta có định lý sau về tính đặt chỉnh của phương trình trên
\begin{theorem}
Phương trình trên là đặt chỉnh khi và chỉ khi $A_{\pi}$ là đẳng cấu và $B$ là toán ánh
\end{theorem}

Giả sử rằng, $V$ và $M$ là không gian Banach phản xạ (reflexive Banach) và $Q = M'$. Bây giờ, ta xét hai dạng song tuyết tính bị chặn $\alpha(V, V)$ và $\beta(V, Q)$, thỏa mãn $\alpha(v, w) =  <Av, w>_{V', V}$ và $b(v, q) = <Bv, q>_{Q', Q}$. Đặt
\begin{equation} \label{eq8}
\begin{split}
||\alpha|| & = \sup_{(v, w) \in V x V} \frac{\alpha(v,w)}{||v||_V||w||_V} \\
||\beta|| & = \sup_{(v, q) \in V x Q} \frac{\beta(v,q)}{||v||_V||q||_Q}
\end{split}
\end{equation}
Với $f \in V'$ và $g \in Q'$. Hệ phương trình có thể viết lại như sau:
\begin{center}
Tìm $u \in V$  và $p \in Q$  thỏa mãn
\end{center}
$$
\begin{cases}
\alpha(u, w) + \beta(w,p) = f(w), \ \forall w \in V \\
\beta(u,q) = g(q), \ \forall q \in Q
\end{cases}
$$
Ở đây, ta kí hiệu $f(v) = <f, v>_{V', V}$ và $g(q) = <g,q>_{Q', Q}$.
Khi đó, ta có định lý sau về tính đặt chỉnh
\begin{theorem}
Hệ phương trình trên đặt chỉnh (well-posed) khi và chỉ khi
\begin{equation} \label{eq8}
\begin{split}
\inf_{v \in ker(B)} \sup_{w \in ker(B)} \frac{\alpha(v,w)}{||v||_V||w||_V} = a > 0 \\
\inf_{q \in Q} \sup_{v \in V} \frac{\beta(v, q)}{||v||_V||q||_Q} = b > 0
\end{split}
\end{equation}
\begin{center}
và $\forall w \in ker(B)$, nếu $\alpha(v, w) = 0$ với mọi $v \in ker(B)$ thì w = 0
\end{center}
\end{theorem}
\newpage
Ở đây, ta thấy rằng, nếu $\alpha(u, v)$ là $V-elliptic$ thì khí đó điều kiện thứ nhất và điều kiện thứ ba dễ dàng được thỏa mãn. Trở lại với bài toán Stoke, ta dễ dàng chứng minh được $\alpha(u, v)$ là $[H_0^1]^2 - elliptic$. Do đó, để bài toán Stoke có nghiệm duy nhất ta chỉ cần phải kiểm tra điều kiên số 2 (điều kiện inf-sup). Và người ta cũng chứng minh được bài toán Stoke được định nghĩa như phần 1 thỏa mãn điều kiện inf-sup.
\chapter{Phương trình Stoke bằng phương pháp phần tử hữu hạn}
\section{Phần tử hữu hạn cho bài toán Stoke}
Trong phần này, ta đi giải bài toán yếu của phương trình stoke bằng phương pháp phần tử hữu hạn, tức là ta đi tìm hàm $u_h$, $v_h$ thuộc vào hai không gian con hữu hạn chiều $V_h$, $Q_h$ của $[ H_0^1(\Omega) ]^2$ và $L_2(\Omega)$ thỏa mãn phương trình
$$
\begin{cases}
\alpha(u_h,v) - \beta(v, p_h) = <f, v> \ \forall \ v \in V_h \\
\beta(u_h, q) = 0 \ \forall \ q \in Q_h
\end{cases}
$$

Theo [1], ta cần phải chọn không gian $V_h$ và $Q_h$ thỏa mãn điều kiện inf-sup
$$\inf_{q_h \in Q_h} \sup_{v_h \in V_h} \frac{\int_{\Omega} q_h \ div \ v_h dx}{||v_h||_{1} ||q_h||_{0/R}}$$

Trong phần này, bài báo cáo sẽ chỉ ra một cách xây dựng hai không gian thỏa mãn điều kiện inf-sup. Ta biết răng, bài toán Stoke thỏa mãn điều kiện inf-sup, theo [2], để kiêm tra điều kiện inf-sup cho hai không gian con hữu hạn chiều $V_h$ và $Q_h$ ta cần xây dựng một ánh xạ $\Pi_h : [H_0^1 (\Omega)]^2 \rightarrow V_h$ sao cho
$$
\begin{cases}
\int_{\Omega} q_h div(\Pi_hv - v) dx = 0 \\
||\Pi_hv|| \leq c ||v||
\end{cases}
$$
Vì $q_h$ có khả vi cấp một, sử dụng công thức tích phân từng phần, ta có điều kiện đầu tương đương với
\begin{equation} \label{eq9}
\begin{split}
\int_{\Omega} (v - \Pi_hv) grad \ q_h dx = 0 \ \forall \ q_h \ \in Q_h
\end{split}
\end{equation}

Ở đây, miền $\Omega$ là một miền đa giác trong không gian hai chiều và được chia thành các phần tử hữu hạn là các tam giác $T$. Và nếu $q_h$ là hàm đa thức bâc $k$ trên mỗi tam giác $T$, điều kiện (9) tương đương với
\begin{equation} \label{eq10}
\begin{split}
\int_{T} (v - \Pi_hv) \Phi_h dx = 0 \ \forall \ \Phi_h \ \in P_{k-1}(T)
\end{split}
\end{equation}
Gọi một cách chia miền $\Omega$ thành các phần tử hữu hạn là $T_h$. Ta định nghĩa
$$M^k(T_h) \ \{ v| v\in C^0(\Omega), v_{|T} \in P_k(T) \ \forall T \in T_h \}$$
$$M_0^k(T_h) = M_0^k(T_h) \cap H_0^1(\Omega)$$
$$B^k(T_h) = \{ v | v_{|T} \in P_k(T) \cap H_0^1(T) \ \forall T \in T_h \}$$

Phương pháp phần tử hữu hạn MINI sử dụng không gian
$$V_h = (M_0^1)^2 \bigoplus (B^3)^2$$
$$Q_h = M_0^1$$

Trong trường hợp này, điều kiện (10) trở thành
\begin{equation} \label{eq11}
\begin{split}
\int_T (v - \Pi_hv) dx = 0 \ \forall \ T, \ \forall \ v \in (H_0^1)^2
\end{split}
\end{equation}

Với cách chọn không gian như trên, ta chỉ cần xây dựng ánh xạ $\Pi_h$ thỏa mãn điều kiện được miêu tả như trên (theo [2]) để chứng minh rằng cách chọn không gian như vậy thỏa mãn điều kiện inf-sup. Theo [3], ta có thể xây dựng được ánh xạ $\overline{\Pi}_h: (H_0^1)^2 \rightarrow (M_0^1)^2$ thỏa mãn (với $r = 0,1$ và $h_T = diam T$)
\begin{equation} \label{eq12}
\begin{split}
\sum_T h_T^{2r-2} ||\overline{\Pi}_hv - v||_{r, T}^2 \leq C ||v||_{1, \Omega}^2
\end{split}
\end{equation}
Để đảm bảo điểu kiện (11), ta cộng thêm vào ánh xạ $\overline{\Pi}_h$ một bội số của hàm bubble trên mỗi tam giác. Cụ thể, ta đặt
\begin{equation} \label{eq13}
\begin{split}
\Pi_hv = \overline{\Pi}_hv + \alpha_T \Phi_T
\end{split}
\end{equation}

Để thỏa mãn điều kiện (11), ta có
\begin{equation} \label{eq14}
\begin{split}
\alpha_T \int_T \Phi_T dx = \int_T (\overline{\Pi}_hv - v)dx
\end{split}
\end{equation}

Ta có
\begin{equation} \label{eq15}
\begin{split}
||\Pi_hv||_{1, T} \leq ||\overline{\Pi}_h||_{1, T} + ||\alpha_T \Phi_T||_{1, T}
\end{split}
\end{equation}

Mặt khác, ta có
\begin{equation} \label{eq16}
\begin{split}
||\alpha_T \Phi_T|| & \leq c ||\alpha_T||\\
|\alpha_T| & \leq ch_T^{-1} ||\Pi_hv - v||_{0, T}
\end{split}
\end{equation}

Lấy tổng theo $T$ và sử dụng (12) ta chứng minh được phương phần tử MINI thỏa mãn điều kiện inf-sup. \\
Trên mỗi tam giác $T$, hàm $u$ và $p$ được xấp xỉ bằng tổ hớp tuyến tính của các hàm cơ sở như sau
\begin{equation} \label{eq17}
\begin{split}
u(\mathbf{x}) = \sum_{i=1}^3 \phi_i(\mathbf{x}) u_i + \phi_b(\mathbf{x}) u_b \\
p(\mathbf{x}) = \sum_{i=1}^3 \phi_i(\mathbf{x})p_i
\end{split}
\end{equation}
trong đó $u_i$ vàn $p_i$ là giá trị nốt của hàm u $u$ và $p$. Phương pháp phần tử hữu hạn $P^1-bubbble / P^1$(MINI) sử dụng hàm các hàm cở được định nghĩa như sau
$$\Phi_1(\mathbf{x}) = 1 - x-y, \ \Phi_2(\mathbf{x}) = x, \ \Phi_3(\mathbf{x}) = y, \ \Phi_b(\mathbf{x}) = 27 \Phi_1(\mathbf{x})\Phi_2(\mathbf{x})\Phi_3(\mathbf{x})$$

Nếu ta đặt
\[
\overline{u}_i
=
\begin{bmatrix}
    u_1^i \\
    u_2^i \\
    u_{i, b}
\end{bmatrix}
, i = 1,2
\]

\[
\overline{f}_i
=
\begin{bmatrix}
    f_1^i \\
    f_2^i \\
    f_{i, b}
\end{bmatrix}
, i = 1,2
\]
Thì hệ phương trình Stoke trên một phần tử hữu hạn có thể viết được dưới dạng như sau,
\begin{equation}
  \begin{bmatrix}
    \overline{A} & 0 & -\overline{B}_1^t \\
    0 & \overline{A} & -\overline{B}_2^t \\
    -\overline{B}_1 & -\overline{B}_2 & 0
  \end{bmatrix}
  \begin{bmatrix}
    \overline{u}_1 \\
    \overline{u}_2 \\
    p
  \end{bmatrix}
  =
  \begin{bmatrix}
    \overline{f}_1 \\
    \overline{f}_2 \\
    0
  \end{bmatrix}
  \label{eq18}
\end{equation}
trong đó
\begin{equation} \label{eq19}
\begin{split}
\overline{A}^T_{ij} & = \int_T \nabla \Phi_i . \nabla \Phi_j dx \\
\overline{B}^T_{ij} & = \int_T \partial_1 \Phi_i \Phi_j dx + \int_T \partial_2 \Phi_i \Phi_j dx \\
\overline{f}^T_i & = \int_T f \Phi_i dx
\end{split}
\end{equation}
Ta cần phải ghép và tính toán các phần tử của ma trận để thu được hệ phương trình trên toàn bộ miền $\Omega$
\section{Các phần tử trên ma trận}
Để giải quyết bài toán, ta chia miền chữ nhật như sau
\begin{center}
\includegraphics[scale=0.5]{grid_fem.PNG}
\end{center}
Với mỗi tam giác $T$, gọi $\{ (x_i, y_i) \}_{i = 1,2,3}$ là các đỉnh và $\{ \Phi_i \}_{i =1,2,3}$ là các hàm cơ sở tương ứng. Khi đó, gradient của $\Phi_i$ được tính bằng công thức sau
\begin{equation} \label{eq20}
\begin{split}
  \begin{bmatrix}
    \nabla \Phi_1^t \\
    \nabla \Phi_2^t \\
	\nabla \Phi_3^t
  \end{bmatrix}
  =
  \frac{1}{2 |T|}
  \begin{bmatrix}
    y_2 - y_3 & x_3 - x_2 \\
    y_3 - y_1 & x_1 - x_3 \\
    y_1 - y_2 & x_2 - x_1
  \end{bmatrix}
\end{split}
\end{equation}
với $|T|$ là diện tích của tam giác $T$, được tính bằng công thức
\begin{equation} \label{eq21}
\begin{split}
2 |T| = det
\begin{bmatrix}
    x_2 - y_1 & x_3 - x_1 \\
    y_2 - y_1 & y_3 - y_1 \\
  \end{bmatrix}
\end{split}
\end{equation}

Ta kí hiệu
$$x_{ij} = x_i - x_j, \ y_{ij} = y_i - y_j$$
và
\[
x^{(T)}
=
\begin{bmatrix}
    x_{32} \\
    x_{13} \\
    x_{21}
\end{bmatrix}
\]

\[
y^{(T)}
=
\begin{bmatrix}
    y_{23} \\
    y_{31} \\
    y_{12}
\end{bmatrix}
\]

Lí do đưa ra hai vector $x^{(T)}$ và $y^{(T)}$ vì nó sẽ được sử dụng để thực hiện lập trình song song trong chương trình Matlab. Khi đó các phần tử của ma trận $\overline{A}$ được tính như sau
\begin{equation} \label{eq22}
\begin{split}
\overline{A}_{ij} = \int_T \nabla \Phi_i \nabla \Phi_j dx = \frac{1}{4|T|} (y_i^{(T)} y_j^{(T)} + x_i^{(T)} x_j^{(T)})
\end{split}
\end{equation}
Khi đó, các phần tử không chứa bubble của ma trận $A$ có thể tính như sau:
$A = \frac{1}{4|T|} \left[ y^{(T)}(y^{(T)})^t + x^{(T)}(x^{(T)})^t \right]$

Vì $\overline{A}$ là ma trận đối xứng, nên ta chỉ cần còn phải tính $\overline{bj}$ với $j = 1,2,3,b$.
\begin{equation} \label{eq23}
\begin{split}
\overline{A}_{bj} =\frac{9|T|}{4} \sum_{i=1}^3 \nabla \Phi_i = 0, \ j =1,2,3
\end{split}
\end{equation}

\begin{equation} \label{eq24}
\begin{split}
\overline{A}_{bb} & = \int_T 27^2 \nabla(\Phi_1 \Phi_2 \Phi_3) \nabla(\Phi_1 \Phi_2 \Phi_3) dx \\
& = \frac{81 |T|}{10} (|\nabla \Phi_1|^2 + |\nabla \Phi_1|^2 + |\nabla \Phi_1|^2 + \nabla \Phi_1\nabla \Phi_2 + \nabla \Phi_2\nabla \Phi_3 + \nabla \Phi_1\nabla \Phi_3) \\
& =: \omega_A
\end{split}
\end{equation}
Như vậy, với kết quả trên, ma trận $\overline{A}$ có dạng
\[
\overline{A} = 
\begin{bmatrix}
A & 0 \\
0 & \omega_A
\end{bmatrix}
\]

Bây giờ, ta đi tính các phần tử của ma trận $\overline{B}_i$. Ta có
\begin{equation} \label{eq25}
\begin{split}
-(\nabla u_h, q_h) & = \overline{B} = \left[ -\overline{B}_1 - \overline{B}_2 \right] \\
& = |T| 
\begin{bmatrix}
-s \nabla \Phi_1 & -s \nabla \Phi_2 & -s \nabla \Phi_3 & t \nabla \Phi_1 \\
-s \nabla \Phi_1 & -s \nabla \Phi_2 & -s \nabla \Phi_3 & t \nabla \Phi_2 \\
-s \nabla \Phi_1 & -s \nabla \Phi_2 & -s \nabla \Phi_3 & t \nabla \Phi_3
\end{bmatrix}
\end{split}
\end{equation}
Trong đó $s = \frac{1}{3}$ và $t = \frac{9}{20}$. Khi đó ta đặt
\[
B_i = \frac{|T|}{3}
\begin{bmatrix}
\partial_i \Phi_1 & \partial_i \Phi_2 & \partial_i \Phi_3 \\
\partial_i \Phi_1 & \partial_i \Phi_2 & \partial_i \Phi_3 \\
\partial_i \Phi_1 & \partial_i \Phi_2 & \partial_i \Phi_3 
\end{bmatrix}
\]
\[
B_{ib} = \frac{9|T|}{20}
\begin{bmatrix}
\partial_i \Phi_1 \\
\partial_i \Phi_2 \\
\partial_i \Phi_3
\end{bmatrix}
\]
Khi đó, ta có
$$\overline{B}_i = |B_i - B_{ib}|$$
và
\begin{equation} \label{eq26}
\begin{split}
B_{1ij} = \frac{|T|}{3} \partial_1 \Phi_i
\end{split}
\end{equation}
Sử dụng các kết quả trên, các phần tử của ma trận $B_1$ và $B_2$ trở thành
\[
B_1 = \frac{1}{6}
\begin{bmatrix}
(y^{(T)})^t \\
(y^{(T)})^t \\
(y^{(T)})^t
\end{bmatrix}
\]
\[
B_2 = \frac{1}{6}
\begin{bmatrix}
(x^{(T)})^t \\
(x^{(T)})^t \\
(x^{(T)})^t
\end{bmatrix}
\]
$$B_{1b} = \frac{9}{40} y^{(T)}, \ $$

Cuối cùng ta chỉ còn phải tính các phần tử của ma trận vế phải. Cụ thể, ta có
\begin{equation} \label{eq27}
\begin{split}
f_i^{(T)} = \frac{|T|}{3} f_{iT}
\begin{bmatrix}
1 \\
1 \\
1
\end{bmatrix}
\end{split}
\end{equation}
trong đó $f_{iT}$ là giá trị trung bình của hàm $f_i$ trên tam giác $T$
$$f_{iT} = (f_i(x_1) + f_i(x_2) + f_i(x_3))$$

\chapter{Kết quả thực nghiệm}
Chương trình Matlab giải phương trình Stoke được miêu tả trong báo cáo có thể download tại \url{https://github.com/Anhtu07/stoke_equation} \\

Các ví dụ được trình bày dưới đây giải bài toán Stoke trên miền $\Omega = [0, 1] \mathbf{x}[0, 1]$. Trong bảng ở trang tiếp theo, \textbf{n} là số phần mà cạnh của hình vuông được chia ra, \textbf{n.o.e} là số phần tử hữu hạn trên miền $\Omega$. \textit{eoc} là được tính bằng loga cơ số 2 của của thương hai số liên tiếp để đánh giá tốc độ của thuật toán. \textit{Totaltime} là thời gian chạy toàn bộ chương trình (bao gồm cả phàn xây dựng ma trận và giải), còn \textit{Solvetime} là thời gian chương trình giải hệ phương trình tuyến tính. Do ma trận được xây dựng không thỏa mãn điều kiện xác định dương, nên việc giải hệ phương trình được thực hiện bằng phương pháp Gauss.
\newpage
\textbf{Ví dụ 1}
\begin{figure}[h!]
\centering
\includegraphics[scale=0.55]{VD1}
\caption{}
\end{figure}

Trong hình 4.2, đồ thị bên trái là nghiệm xấp xỉ thu được, còn đồ thị bên phải là nghiệm đúng của phương trình. \\

Để quan sát được rõ hơn độ sai lệch giữa nghiệm tìm được và nghiệm đúng của $u_1$, hình 4.3 biểu diễn đồ thị hiệu hai nghiệm. Nghiệm gần đúng được tính sử dụng 7938 phần tử hữu hạn \\
\begin{figure}[h!]
\centering
\includegraphics[scale=0.8]{450_tri_1}
\caption{ Nghiệm xấp xỉ và Nghiệm đúng}
\end{figure}

\begin{figure}[h!]
\centering
\includegraphics[scale=0.7]{sub1}
\caption{ Hiệu hai nghiệm}
\end{figure}
\newpage
Trong bảng (hình 4.1) ta có chú rằng sai số của $L_2$ của $u_1$ và $u_2$ bằng nhau. Điều này là do hai hàm $u_1$, $u_2$ đối xứng với nhau. Ở trong ví dú 2, ta xem xét hai hàm không đối xứng với nhau. \\

\textbf{Ví dụ 2}
\begin{figure}[h!]
\centering
\includegraphics[scale=0.55]{VD2}
\caption{}
\end{figure}

Các hình tiếp theo biểu diễn nghiệm xấp xỉ và nghiệm đúng của $u_2$ thu được khi chạy ví dụ 2.
\begin{figure}[h!]
\centering
\includegraphics[scale=0.6]{450_tri_2}
\caption{}
\end{figure}
\begin{figure}[h!]
\centering
\includegraphics[scale=0.7]{sub2}
\caption{}
\end{figure}
\newpage

Cuối cùng, ta có nhận xét về sai số $L_2$ đối với phương pháp phần tử hữu hạn sử dụng $P_1-bubble / P_1$ rằng sai số của của $u$ của $p$ tuân theo kết quả chứng minh được từ lý thuyết với tốc độ hội tụ của $u$ là $O(h^2)$ còn của $p$ là $O(h^{3/2}$. Còn về sai số điểm, nhóm làm báo cáo chưa tìm thấy kết quả về vấn đề này, và thực nghiệm thì không cho ra được dự đoán nào về tốc độ hội tụ của sai số điểm. \\[1cm]

{\huge \textbf{Tài liệu tham khảo}} \\
$[1]$ D.N Arnold, F. Brezzi, M. Fortin, A stable finite element for the Stokes Equations \\
$[2]$ F. Brezzi, On the existence, uniqueness and approximation of  saddle point problems arising from Lagrangian multipliers \\
$[3]$ Janas Koko, Stokes problem with $P^1-bubble/P^1$ Finite Element\\
$[4]$ Alexandre Ern, Jean-Luc Guermond, Theory and Practice of Finite Elements
\end{document}
